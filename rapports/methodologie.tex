\chapter{Méthodologie}     % numéroté
\label{chap:methodologie}                   % étiquette pour renvois (à compléter!)

Le coeur de notre travail consiste à proposer une méthode de détection d'anomalies où nous utilisons la représentation latente d'un autoencodeur variationnel qui sera transformée vers une métrique que nous pourrons tester via un test d'hypothèse. Ce test d'hypothèse nous permet donc de détecter des anomalies avec un niveau de confiance.

\section{Les hypothèses de l'approche}

Dans notre approche, nous supposons que  nous avons accès à un jeu de données qui contient presque qu'entièrement des observations normales. En d'autres mots, il n'y pratiquement pas d'anomalies. Par contre, nous anticipons des entrées anormales dans le futur et nous voulons les détecter. Voici 2 exemples de situation où ce genre de contexte pourrait être plausible:

\begin{enumerate}
	\item Une compagnie d'assurance propose maintenant à leurs assurés de prendre eux-mêmes des photos de leur véhicule accidenté et d'envoyer celles-ci à l'assureur. La compagnie s'attend à ce que les assurés puissent commettre des erreurs de manipulation en prenant les photos ou bien puissent envoyer les mauvaises photos. De son côté, l'assureur possède des photos de véhicules accidentés, prises par son personnel d'estimateurs, mais aucune photo erronée. Le jeu de données d'entraînement ne possède théoriquement aucune anomalie.
	\item Une entreprise manufacturière qui produit des pièces de voiture vient de se procurer un nouveau système qui automatise davantage la production. L'entreprise veut faire un suivi des pièces produites par ce nouveau système via une caméra qui inspecte les produits finaux. L'entreprise possède déjà plusieurs images de pièces qui étaient produites avant l'acquisition du nouveau système. Le résultat final produit par ce nouveau système doit être identique à l'ancien. Par contre, il est difficile de prévoir de quoi auront l'air des défectuosités produites par ce nouveau système. Le jeu de données d'entraînement est donc constitué de plusieurs exemples normales, mais aucune anomalie.
\end{enumerate}

\section{Description de l'approche}

Considérant que nous avons accès à un jeu de données ayant les caractéristiques décrites à la section précédente, nous proposons d'entraîner un autoencoder variationnel pour apprendre les caractéristiques, ou la distribution, de cette population "normale". Cette distribution apprise sera essentiellement contenu dans la représentation latente du VAE entraîné, ou plus spécifiquement dans les couches $\mu$ et $\sigma$ du réseau. Comme nous l'avons vu dans la section \ref{background-vae}, les VAE ont la particularité d'avoir une composante de perte associée à cette représentation latente, s'assurant ainsi que celle-ci s'approche d'une distribution à priori, soit une $N(0, I)$ dans notre cas. Une fois que l'autoencodeur est entraîné, nous pouvons utiliser la partie encodeur du réseau pour transformer chacune des instances de notre jeu de données d'entraînement vers des  vecteurs $\mu$ et $\sigma$. Ces représentations latentes, encodées sous forme de vecteurs, contiennent l'information qui devrait permettre de savoir si une nouvelle instance partage ou non cette même information. Pour parvenir à faire cette conclusion sur une nouvelle instance, nous devons comparer l'information contenue de la représentation latente de cette instances avec toutes les représentations latentes de l'ensemble d'entraînement. Il est donc plus pratique d'agréger notre population d'entraînement en une seule représentation globale. Une première manière d'y arriver est de calculer un vecteur $\bar{\mu}$ et un vecteur $\bar{\sigma}$, qui sont simplement une moyenne sur la population d'entraînement. Supposons que notre jeu de données d'entraînement contient $n$ observations et que nous encodons celles-ci vers des représentations latentes à $k$ dimensions, l'élément $i$ des vecteurs $\bar{\mu}$ et $\bar{\sigma}$ sera donné par :

\begin{gather*}
\bar{\mu}^{(i)} = \frac{1}{n} \sum_{k=1}^{n} \mu^{(k)} \\
\bar{\sigma}^{(i)} = \frac{1}{n} \sum_{k=1}^{n} \sigma^{(k)}
\end{gather*}

Étant donné que les vecteurs $\bar{\mu}$ et $\bar{\sigma}$ sont essentiellement composé d'instances normales, on peut s'attendre que les vecteurs $\mu$ et $\sigma$ de nouvelles instances normales soient relativement près de ces vecteurs moyen. À l'inverse, on peut s'attendre que ces 2 mêmes vecteurs, pour des instances anormales, ou des anomalies, soient plutôt loin de ces vecteurs moyens. Maintenant, comment savoir si la représentation latente d'une nouvelle instance est proche ou éloignée de cette représentation agrégée? Étant donné que les vecteurs $\mu$ et $\sigma$ représentent les paramètres d'une loi normale multivariée, nous pouvons utiliser la distance de Kullbach-Leibler pour quantifier le degré de similitude entre la distribution moyenne et la distribution $\mu$ et $\sigma$ de la nouvelle instance. Cette même distance est d'ailleurs utilisée lors de l'optimisation pour calculer la composante de perte associée à la représentation latente. Finalement, il est possible d'utiliser les distances de KL calculées sur l'ensemble d'entraînement pour savoir si une nouvelle distance est élevée ou non. Cela devient possible étant donné que notre ensemble d'entraînement contient presque uniquement des observations normales, donc des distances de KL qui devraient être petites. On peut d'ailleurs utiliser ces distances ordonnées pour avoir une mesure de probabilité d'une nouvelle instance. La méthodologie proposée pour calculer cette mesure de probabilité, ou cette valeur-$p$ est résumé dans l'algorithme \ref{anomaly_algo}.
\newline


\begin{center}
	\begin{algorithm}[H] \label{anomaly_algo}
		\SetAlgoLined
		\KwIn{Ensemble de données d'entraînement sans anomalie $x^{(j)}, j=1,...,m$, \\ Ensemble de données de test avec anomalies $y^{(i)}, i=1, ..., n$}
		\KwOut{Valeurs-$p$ pour chaque instance de test $p^{(i)}, i=1,...,n$}
		$\theta$, $\phi$ $\leftarrow$ paramètres de l'encodeur ($q_{\theta}(x)$) et du  décodeur ($p_{\phi}(z)$) du VAE entraîné\;
		\For{j=1 to m}{
			$\mu^{(j)} = p_{\theta}(x^{(j)})["mu"]$\;
			$\sigma^{(j)} = p_{\theta}(x^{(j)})["sd"]$\;
		}
		$\bar{\mu} = \frac{1}{m} \sum_{k=1}^{m} \mu^{(k)}$\;
		$\bar{\sigma} = \frac{1}{m} \sum_{k=1}^{m} \sigma^{(k)}$\;
		\For{j=1 to m}{
			$kl\_train^{(j)}=kl\_distance(\mu^{(j)}, \sigma^{(j)}, \bar{\mu}, \bar{\sigma})$
		}
		$kl\_sorted = sort(kl\_train)$\;
		\For{i=1 to n}{
			$\mu^{(i)} = p_{\theta}(y^{(i)})["mu"]$\;
			$\sigma^{(i)} = p_{\theta}(y^{(i)})["sd"]$\;
			$kl\_test^{(j)}=kl\_distance(\mu^{(i)}, \sigma^{(i)}, \bar{\mu}, \bar{\sigma})$\;
			$p^{(i)} = rank_{kl\_sorted}(kl\_test)$ / m
		}
		\KwRet{$\boldsymbol{p}$}
		\caption{Algorithme de détection d'anomalies basé sur un autoencodeur variationnel}
	\end{algorithm}
\end{center}

Dans l'approche que nous proposons, la valeur-$p$ est calculée à partir d'une distribution empirique, soit les distances de Kullbach-Leibler de toutes les instances du jeu de données d'entraînement. Au final, nous testons empiriquement si une nouvelle observation semble provenir de la population d'entraînement:

\begin{center}
	$\boldsymbol{H_0}$: $y^{(i)}$ provient de la population $X$ \\
	$\boldsymbol{H_1}$: $y^{(i)}$ ne provient pas de la population $X$
\end{center}

\noindent Puisque nous supposons que $X$, soit l'ensemble de données d'entraînement, est composé en grande partie d'observations normales, nous pouvons réécrire notre test comme étant:

\begin{center}
	$\boldsymbol{H_0}$: $y^{(i)}$ n'est pas une anomalie \\
	$\boldsymbol{H_1}$: $y^{(i)}$ est une anomalie
\end{center}

Finalement, une fois que nous avons notre test statistique de définit, nous pouvons utiliser les valeurs-$p$ calculées sur l'ensemble de test et conclure avec un niveau de confiance $\alpha$ si une instance est une anomalie (voir algorithme \ref{test_algo}).

\begin{center}
	\begin{algorithm}[H] \label{test_algo}
		\SetAlgoLined
		\KwIn{Valeurs-$p$ pour les instances de test  $p^{(i)}, i=1,...,n$, \\ niveau de confiance $\alpha$}
		\KwOut{indicateurs d'anomalies $o^{(i)}, i=1,...,n$}
		\For{i=1 to n}{
			\eIf{$p^{(i)} < \alpha$}{
				$o^{(i)}$ = $vrai$
			}
			{
				$o^{(i)}$ = $faux$
			}
		}
		\KwRet{$\boldsymbol{o}$}
		\caption{Algorithme de prise de décision}
	\end{algorithm}
\end{center}

\section{Approche pour des données complexes}

Parler de pourquoi utiliser les representation encoder pour le test au lieu de l'erreur de reconstruction. Peut-etre parler de la perceptual loss pour dealer avec une reconstruction difficile pour des images complex.

\subsection{Représentation latentes}

\subsection{Perceptual loss}

\section{Avantages du test d'hypothèse}

Parler du fait que c'est simple de definir un level of significance versus trouver une metric quelconque.