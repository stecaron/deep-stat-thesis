\chapter*{Abstract}             % ne pas numéroter
\label{chap:abstract}           % étiquette pour renvois
\phantomsection\addcontentsline{toc}{chapter}{\nameref{chap:abstract}} % inclure dans TdM

\begin{otherlanguage*}{english}
In this paper, we propose a methodology that aims to detect anomalies among complex data, such as images, and that is based on a confidence level rather than on a problem-specific threshold or value. In order to do that, we demonstrate the usefulness of using variational autoencoders (VAE) to deal with complex data and also to have some kind of representation from which we can apply hypothesis testings. From our experiments, we can show that our approach is able to detect images that are outliers in a given dataset by only specifying a certain confidence level. By using this approach, the anomaly detection becomes feasible for real-world complex images and is also easier to maintain and interpret considering the detection is made on a confidence level rather than on a vague and possibly changing metric.
\end{otherlanguage*}
