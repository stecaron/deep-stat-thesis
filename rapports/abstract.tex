\chapter*{Abstract}             % ne pas numéroter
\label{chap:abstract}           % étiquette pour renvois
\phantomsection\addcontentsline{toc}{chapter}{\nameref{chap:abstract}} % inclure dans TdM

\begin{otherlanguage*}{english}
In this master's thesis, we propose a methodology that aims to detect anomalies among complex data, such as images. In order to do that, we use a specific type of neural network called the varitionnal autoencoder (VAE). This non-supervised deep learning approach allows us to obtain a simple representation of our data on which we then use the Kullback-Leibler distance to discriminate between anomalies and "normal" observations. To determine if an image should be considered "abnormal", our approach is based on a proportion of observations to be filtered, which is easier and more intuitive to establish than applying a threshold based on the value of a distance metric. By using our methodology on real complex images, we can obtain superior anomaly detection performances in terms of area under the ROC curve (AUC), precision and recall compared to other non-supervised methods. Moreover, we demonstrate that the simplicity of our filtration level allows us to easily adapt the method to datasets having different levels of anomaly contamination.
\end{otherlanguage*}