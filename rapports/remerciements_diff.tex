\chapter*{Remerciements}        % ne pas numéroter
\label{chap:remerciements}      % étiquette pour renvois
\phantomsection\addcontentsline{toc}{chapter}{\nameref{chap:remerciements}} % inclure dans TdM

Je tiens tout d'abord à remercier mon directeur Thierry Duchesne pour son support constant tout au long de mon mémoire. Ses conseils et ses commentaires m'ont permis d'avancer plus rapidement et également d'en apprendre beaucoup sur la rédaction de documents scientifiques. Je le remercie d'avoir fait preuve de patience dans l'écriture de ce mémoire que j'ai décidé de rédiger à temps partiel, s'étirant ainsi sur une plus longue période de temps. Aussi, je souhaite remercier mon conseiller en recherche, collègue et ami François-Michel pour tous ses conseils ingénieux qui m'ont permis de résoudre plusieurs problèmes techniques au cours de mes expérimentations. Dans \DIFdelbegin \DIFdel{a }\DIFdelend \DIFaddbegin \DIFadd{la }\DIFaddend pratique, François-Michel a joué le rôle de "codirecteur" dans ce projet. Je souhaite également remercier mon collègue départemental, mais surtout mon ami très proche, Samuel, pour toutes les heures passées au téléphone à m'écouter partager mes idées, mes problèmes techniques ou mes états d'âme. Sam a été un mentor pour moi tout au long de ma maîtrise et de ce projet. Je me considère extrêmement chanceux d'avoir un ami comme lui, qui ne refuse pratiquement jamais de m'aider, et ce, peu importe l'heure et la journée ... 

Par ailleurs, je tiens aussi à remercier mon employeur Intact Assurance de m'avoir supporté tout au long de ma maîtrise que j'ai faite à temps partiel.

Je ne pourrais pas terminer des remerciements sans remercier ma famille et mon amoureuse Violaine. Mes parents, qui m'ont épaulé tout au long de mon parcours scolaire, sont à mes yeux un facteur déterminant dans la réalisation de ce projet. Finalement, mon amoureuse Violaine a également été une pièce maîtresse dans la réussite de ce mémoire. Elle a été le facteur le plus important dans la préservation de mon moral, qui a été mis à l'épreuve à plusieurs reprises pour achever ce mémoire. Elle m'a soutenu de mille et une manières, ce qui m'a grandement poussé à mettre les efforts nécessaires. Elle est une source de motivation pour moi et c'est d'ailleurs elle qui m'a donné l'énergie et la force dans le dernier droit de ce projet qui sera enrichissant et structurant pour le reste de ma carrière. Merci à vous tous ! 

