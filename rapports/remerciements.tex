\chapter*{Remerciements}        % ne pas numéroter
\label{chap:remerciements}      % étiquette pour renvois
\phantomsection\addcontentsline{toc}{chapter}{\nameref{chap:remerciements}} % inclure dans TdM

\begin{comment}
Je tiens tout d'abord à remercier mon directeur Thierry Duchesne pour son support constant tout au long de mon mémoire. Ses conseils et ses commentaires m'ont permis d'avancer plus rapidement et également d'apprendre beaucoup la rédaction de documents scientifiques. Aussi, je le remercie d'avoir fait preuve de patience dans l'écriture de ce mémoire que j'ai décidé de le rédiger à temps partiel, s'étirant ainsi sur une plus longue période de temps. Aussi, je souhaite remercier mon co-directeur, collègue et ami François-Michel pour tous ses conseils ingénieux qui m'ont permis de résoudre plusieurs problèmes techniques tout au long de mes expérimentations. Je souhaite également remercier mon collègue départementale, mais surtout mon ami très précieux Samuel pour toutes les heures passées au téléphone à m'écouter ventiler sur des problèmes techniques rencontrés dans ce mémoire. Sam m'a servi de boussole dans ce projet. Je me considère vraiment chanceux d'avoir un ami comme lui, qui refuse pratiquement jamais d'aider, et ce, peu importe l'heure et la journée ... 

Par ailleurs, je tiens aussi à remercier mon employeur Intact Assurance de m'avoir supporté tout au long de ma maîtrise que je faisais à temps partiel. 

Finalement, je ne pourrais pas terminer des remerciements sans remercier ma famille et mon amoureuse Violaine. Mes parents, qui m'ont épaulé tout au long de mon parcours scolaire, sont un facteur implicite et déterminant dans la réalisation de ce projet. Mon amoureuse Violaine est quant à elle une pièce maîtresse dans la préservation de mon moral, qui a été mis à l'épreuve à plusieurs reprises pour achever ce mémoire. Je te remercie d'avoir endurer ces nombreuses heures passées devant un écran. Elle est d'ailleurs le moteur qui m'a donner l'énergie et la force de finir ce projet enrichissant et structurant pour le reste de ma carrière.
\end{comment}

