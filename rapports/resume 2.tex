\chapter*{Résumé}             % ne pas numéroter
\label{chap:resume}           % étiquette pour renvois
\phantomsection\addcontentsline{toc}{chapter}{\nameref{chap:resume}} % inclure dans TdM


Dans ce mémoire, nous proposons une méthodologie qui permet de détecter des anomalies parmi un ensemble de données complexes, plus particulièrement des images. Pour y arriver, nous utilisons un type spécifique de réseau de neurones, soit un autoencodeur variationnel (VAE). Cette approche non-supervisée d'apprentissage profond nous permet d'obtenir une représentation plus simple de nos données sur laquelle nous appliquerons une mesure de distance de Kullback-Leibler nous permettant de discriminer les anomalies des observations "normales". Pour déterminer si une image nous apparaît comme "anormale", notre approche se base sur une proportion d'observations à filtrer, ce qui est plus simple et intuitif à établir qu'un seuil sur la valeur même de la distance. En utilisant notre méthodologie sur des images réelles, nous avons démontré que nous pouvons obtenir des performances de détection d'anomalies supérieures en termes d'aire sous la courbe ROC, de précision et de rappel par rapport à d'autres approches non-supervisées. De plus, nous avons montré que la simplicité de l'approche par niveau de filtration permet d'adapter facilement la méthode à des jeux de données ayant différents niveaux de contamination d'anomalies.
