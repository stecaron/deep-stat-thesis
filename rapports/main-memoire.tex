%% GABARIT POUR MÉMOIRE STANDARD
%%
%% Consulter la documentation de la classe ulthese pour une
%% description détaillée de la classe, de ce gabarit et des options
%% disponibles.
%%
%% [Ne pas hésiter à supprimer les commentaires après les avoir lus.]
%%
%% Déclaration de la classe avec le type de grade
%%   [l'un de MATDR, MArch, MA, LLM, MErg, MMus, MPht, MSc, MScGeogr,
%%    MServSoc, MPsEd]
%% et les langues les plus courantes. Le français sera la langue par
%% défaut du document.
\documentclass[MSc,english,french]{ulthese}
  %% Encodage utilisé pour les caractères accentués dans les fichiers
  %% source du document. Les gabarits sont encodés en UTF-8. Inutile
  %% avec XeLaTeX, qui gère Unicode nativement.
  \ifxetex\else \usepackage[utf8]{inputenc} \fi

  %% Charger ici les autres paquetages nécessaires pour le document.
  %% Quelques exemples; décommenter au besoin.
  \usepackage{amsmath}       % recommandé pour les mathématiques
  \usepackage{icomma}        % gestion de la virgule dans les nombres

  %% Utilisation d'une autre police de caractères pour le document.
  %% - Sous LaTeX
  %\usepackage{mathpazo}      % texte et mathématiques en Palatino
  %\usepackage{mathptmx}      % texte et mathématiques en Times
  %% - Sous XeLaTeX
  %\setmainfont{TeX Gyre Pagella}      % texte en Pagella (Palatino)
  %\setmathfont{TeX Gyre Pagella Math} % mathématiques en Pagella (Palatino)
  %\setmainfont{TeX Gyre Termes}       % texte en Termes (Times)
  %\setmathfont{TeX Gyre Termes Math}  % mathématiques en Termes (Times)
  
  %DIF PREAMBLE EXTENSION ADDED BY LATEXDIFF
  %DIF UNDERLINE PREAMBLE %DIF PREAMBLE
  \RequirePackage[normalem]{ulem} %DIF PREAMBLE
  \RequirePackage{color}\definecolor{RED}{rgb}{1,0,0}\definecolor{BLUE}{rgb}{0,0,1} %DIF PREAMBLE
  \providecommand{\DIFadd}[1]{{\protect\color{blue}\uwave{#1}}} %DIF PREAMBLE
  \providecommand{\DIFdel}[1]{{\protect\color{red}\sout{#1}}}                      %DIF PREAMBLE
  %DIF SAFE PREAMBLE %DIF PREAMBLE
  \providecommand{\DIFaddbegin}{} %DIF PREAMBLE
  \providecommand{\DIFaddend}{} %DIF PREAMBLE
  \providecommand{\DIFdelbegin}{} %DIF PREAMBLE
  \providecommand{\DIFdelend}{} %DIF PREAMBLE
  %DIF FLOATSAFE PREAMBLE %DIF PREAMBLE
  \providecommand{\DIFaddFL}[1]{\DIFadd{#1}} %DIF PREAMBLE
  \providecommand{\DIFdelFL}[1]{\DIFdel{#1}} %DIF PREAMBLE
  \providecommand{\DIFaddbeginFL}{} %DIF PREAMBLE
  \providecommand{\DIFaddendFL}{} %DIF PREAMBLE
  \providecommand{\DIFdelbeginFL}{} %DIF PREAMBLE
  \providecommand{\DIFdelendFL}{} %DIF PREAMBLE
  %DIF END PREAMBLE EXTENSION ADDED BY LATEXDIFF
  
 \usepackage{graphicx}
 \usepackage{subcaption}
 \usepackage{amsfonts}
 \usepackage{float} 
  
 % Section reference
 \usepackage{hyperref}

 % Networks
 \def\layersep{2cm}
 \usepackage{tikz}
 \usepackage{verbatim}
 \usetikzlibrary{decorations.pathreplacing}
 \usetikzlibrary{decorations.pathreplacing}
 \usetikzlibrary{shapes.geometric}
	
 %Tableau
 \usepackage{array,multirow,makecell}
 \usepackage{xcolor,colortbl}
 \definecolor{Gray}{gray}{0.85}
 \newcolumntype{a}{>{\columncolor{white}}c}
 \setlength{\aboverulesep}{0pt}
 \setlength{\belowrulesep}{0pt}
 
 \newcommand{\PLH}{{\mkern-2mu\times\mkern-2mu}}
 \newcommand\norm[1]{\left\lVert#1\right\rVert}

% Algorithms
\usepackage[ruled,vlined]{algorithm2e}

  %% Options de mise en forme du mode français de babel. Consulter la
  %% documentation du paquetage babel pour les options disponibles.
  %% Désactiver (effacer ou mettre en commentaire) si l'option
  %% 'nobabel' est spécifiée au chargement de la classe.
  \frenchbsetup{%
    StandardItemizeEnv=true,       % format standard des listes
    ThinSpaceInFrenchNumbers=true, % espace fine dans les nombres
    og=«, fg=»                     % caractères « et » sont les guillemets
  }

  %% Style de la bibliographie.
 \bibliographystyle{apalike}

  %% Composition de la page frontispice. Remplacer les éléments entre < >.
  %% Supprimer les caractères < >. Couper un long titre ou un long
  %% sous-titre manuellement avec \\.
  \titre{Détection d'anomalies basée sur les représentations latentes d'un autoencodeur variationnel}
  % \titre{Ceci est un exemple de long titre \\
  %   avec saut de ligne manuel}
  % \soustitre{Sous-titre le cas échéant}
  % \soustitre{Ceci est un exemple de long sous-titre \\
  %   avec saut de ligne manuel}
  \auteur{Stéphane Caron}
  \programme{Maîtrise en statistique - avec mémoire}
  \direction{Thierry Duchesne, directeur de recherche}
  \codirection{François-Michel De Rainville, conseiller en recherche \\
  						Samuel Perreault, conseiller en recherche}
  %              <Prénom Nom>, <codirecteur ou codirectrice> de recherche}

  %% Les commandes ci-dessous servent uniquement pour la création
  %% d'une page de titre (interdite lors du dépôt à la FESP).
  \annee{2020}

\begin{document}

\frontmatter                    % pages liminaires
\pagetitre                    % production de la page frontispice

\chapter*{Résumé}             % ne pas numéroter
\label{chap:resume}           % étiquette pour renvois
\phantomsection\addcontentsline{toc}{chapter}{\nameref{chap:resume}} % inclure dans TdM


Dans ce mémoire, nous proposons une méthodologie qui permet de détecter des anomalies parmi un ensemble de données complexes, plus particulièrement des images. Pour y arriver, nous utilisons un type spécifique de réseau de neurones, soit un autoencodeur variationnel (VAE). Cette approche non-supervisée d'apprentissage profond nous permet d'obtenir une représentation plus simple de nos données sur laquelle nous appliquerons une mesure de distance de Kullback-Leibler nous permettant de discriminer les anomalies des observations "normales". Pour déterminer si une image nous apparaît comme "anormale", notre approche se base sur une proportion d'observations à filtrer, ce qui est plus simple et intuitif à établir qu'un seuil sur la valeur même de la distance. En utilisant notre méthodologie sur des images réelles, nous avons démontré que nous pouvons obtenir des performances de détection d'anomalies supérieures en termes d'aire sous la courbe (ROC), de précision et de rappel par rapport à d'autres approches non-supervisées. De plus, nous avons montré que la simplicité de l'approche par niveau de filtration permet d'adapter facilement la méthode à des jeux de données ayant différents niveaux de contamination d'anomalies.
                % résumé français
\chapter*{Abstract}             % ne pas numéroter
\label{chap:abstract}           % étiquette pour renvois
\phantomsection\addcontentsline{toc}{chapter}{\nameref{chap:abstract}} % inclure dans TdM

\begin{otherlanguage*}{english}
In this paper, we propose an anomaly detection methodology that aims to detect anomalies among complex data, such as images, and that is based on a confidence level rather than on a certain threshold or metric. In order to do that, we demonstrate the usefulness of using variational autoencoders (VAE) to deal with complex data and also to have some kind of representation from which we can apply hypothesis testings. From our experiments, we can show that our approach is able to detect images that are outliers in a given dataset by only specifying a certain confidence level. By using this approach, the anomaly detection becomes feasible for real-world complex images and is also easier to maintain and interpret considering the detection is made on a confidence level rather than on a vague and possibly changing metric.
\end{otherlanguage*}
              % résumé anglais
\cleardoublepage

\chapter*{Remerciements}        % ne pas numéroter
\label{chap:remerciements}      % étiquette pour renvois
\phantomsection\addcontentsline{toc}{chapter}{\nameref{chap:remerciements}} % inclure dans TdM

Je tiens tout d'abord à remercier mon directeur Thierry Duchesne pour son support constant tout au long de mon mémoire. Ses conseils et ses commentaires m'ont permis d'avancer plus rapidement et également d'en apprendre beaucoup sur la rédaction de documents scientifiques. Je le remercie d'avoir fait preuve de patience dans l'écriture de ce mémoire que j'ai décidé de rédiger à temps partiel, s'étirant ainsi sur une plus longue période de temps. Aussi, je souhaite remercier mon co-directeur, collègue et ami François-Michel pour tous ses conseils ingénieux qui m'ont permis de résoudre plusieurs problèmes techniques au cours de mes expérimentations. Je souhaite également remercier mon collègue départementale, mais surtout mon ami très proche, Samuel, pour toutes les heures passées au téléphone à m'écouter partager mes idées, mes problèmes techniques ou mes états d'âme. Sam a été un mentor pour moi tout au long de ma maîtrise et de ce projet. Je me considère extrêmement chanceux d'avoir un ami comme lui, qui ne refuse pratiquement jamais de m'aider, et ce, peu importe l'heure et la journée ... 

Par ailleurs, je tiens aussi à remercier mon employeur Intact Assurance de m'avoir supporté tout au long de ma maîtrise que j'ai fait à temps partiel.

Je ne pourrais pas terminer des remerciements sans remercier ma famille et mon amoureuse Violaine. Mes parents, qui m'ont épaulé tout au long de mon parcours scolaire, sont à mes yeux un facteur déterminant dans la réalisation de ce projet. Finalement, mon amoureuse Violaine a également été une pièce maîtresse dans la réussite de ce mémoire. Elle a été un facteur important dans la préservation de mon moral, qui a été mis à l'épreuve à plusieurs reprises pour achever ce mémoire. Elle m'a soutenu de mille et une manières, ce qui m'a grandement aidé à mettre les efforts nécessaires. Elle est une source de motivation pour moi et c'est d'ailleurs elle qui m'a donné l'énergie et la force dans le dernier droit de ce projet qui sera enrichissant et structurant pour le reste de ma carrière. Merci à vous tous ! 

         % remerciements

\tableofcontents                % production de la TdM
\cleardoublepage

\listoftables                   % production de la liste des tableaux
\cleardoublepage

\listoffigures                  % production de la liste des figures
\cleardoublepage


%\epigraphe{<Texte de l'épigraphe>}{<Source ou auteur>}
%\cleardoublepage

\mainmatter                     % corps du document

\chapter*{Introduction}         % ne pas numéroter
\label{chap:introduction}       % étiquette pour renvois
\phantomsection\addcontentsline{toc}{chapter}{\nameref{chap:introduction}} % inclure dans TdM


La détection d'anomalie est un sujet complexe qui a généré beaucoup de littérature en statistique,  en apprentissage machine et plus récemment en vision numérique. Il existe plusieurs applications à ce sujet dans les domaines de la cyber-intrusion, de la finance et de l'assurance, de la médecine ou dans l'identification de dommages industriels (\cite{Chandola07anomalydetection:}). Une anomalie est définie par \cite{Zimek2017} comme un événement, une mesure ou une observation qui diffère significativement de la majorité des données. Une anomalie, ou également appelée une aberration, est un concept intrinsèque à plusieurs domaines reliés à l'analyse de données, car une telle donnée est généralement intéressante à identifier ou retirer d'une source de données. Il peut être intéressant de l'identifier simplement parce que c'est  la tâche poursuivie. Il peut également être intéressant de la retirer avant de réaliser une autre tâche d'apprentissage. \newline

Une difficulté reliée à la détection d'anomalie est qu'on doit souvent utiliser des données non étiquetées, car les anomalies sont généralement générées par des phénomènes imprévus. Cela fait en sorte que le problème devient non-supervisé. Une autre difficulté, plus particulièrement reliée avec les approches non-supervisées, est qu'il faut généralement déterminer un seuil qui nous permet de prendre une décision quant à la nature d'une donnée (anomalie ou non). Finalement, ces difficultés deviennent encore plus prononcées lorsqu'on doit traiter des données complexes, comme des images. \newline

Dans le contexte de données à hautes dimensions, les réseaux de neurones sont couramment utilisés. En effet, leurs couches superposées peuvent partir d'une entrée complexe, comme une image, et compresser cette information vers une représentation vectorielle plus simple et riche en informations. Les autoencodeurs sont une catégorie de réseaux de neurones fréquemment utilisés pour traiter un problème non-supervisé. Il existe d'ailleurs plusieurs applications d'autoencodeurs dans un contexte de détection d'anomalie, où l'erreur de reconstruction est souvent utilisée comme indicateur d'anomalie. Cependant, ces méthodes requièrent de trouver un seuil, souvent sous forme de distance ou de métrique, qui peut être difficile à établir ou à expliquer. C'est d'ailleurs pour cette raison que  \cite{An2015VariationalAB} proposent de se baser sur une probabilité de reconstruction, qui est un mesure plus objective et ne requiert pas de seuil quelconque. Par contre, cette mesure de probabilité est basée sur la capacité de reconstruction, qui peut être problématique dans le contexte d'images complexes, plutôt que sur la représentation latente, qui elle est plus simple.  \newline

Dans cette étude, nous proposons une approche qui vise à simplifier la détermination du seuil nécessaire pour la détection d'anomalie à partir de données non étiquetées. Avec cette contribution, nous proposons d'utiliser des méthodes existantes comme les autoencodeurs et les tests d'hypothèses pour encoder des structures de données complexes dans un cadre décisionnel simple et intuitif. À partir d'autoencodeurs variationnels (VAE), nous sommes en mesure de créer ces dites représentations et de détecter les anomalies à partir d'un niveau de confiance, plutôt que d'une métrique ou une distance.          % introduction
\chapter{Contexte}     % numéroté
\label{chap:background}                   % étiquette pour renvois (à compléter!)

En premier lieu, nous allons commencer par faire une brève revue des différentes catégories  de méthodes de détection d'anomalies et de leur fonctionnement respectif. Ensuite, nous allons faire un résumé de la théorie des autoencodeurs et expliquer comment ceux-ci sont pertinents dans un contexte de détection d'anomalies. Ensuite, nous allons décrire plus en détails un type particulier d'autoencodeur utilisé dans cette étude, soit l'autoencodeur variationnel. Finalement, nous allons expliquer comment les autoencodeurs, plus particulièrement les autoencodeurs variationnels, peuvent être utilisés dans un contexte de détection d'anomalies.

\section{Les méthodes de détection d'anomalies}

Tout d'abord, il est pertinent de commencer par mentionner que toutes les méthodes de détection d'anomalies fonctionnent fondamentalement de la même manière. En effet, ces algorithmes sont en mesure de faire l'apprentissage d'un jeu de données et de stocker cette information dans un modèle d'apprentissage. En sachant ce qui est "normal", il est donc également possible d'utiliser ce modèle pour évaluer ce qui est "anormal" en identifiant ce qui dévie de cette normalité. Le choix du modèle est crucial, car si celui-ci ne s'ajuste pas bien aux données, il pourrait nous induire en erreur sur l'anormalité d'une observation \citep{10.5555/3086742}. 

Pour démontrer l'importance du choix de modèle, prenons un exemple simpliste et fréquemment utilisé en pratique, soit sur la moyenne d'une loi normale à variance connue, souvent appelé le test $Z$. Dans ce test statistique, qui peut être utilisé pour de la détection d'anomalies, l'hypothèse nulle est que les données suivent une loi normale. Prenons par exemple des observations à une dimension $X_1, ..., X_N$ de moyenne $\mu$ et d'écart-type $\sigma$. La valeur $Z$ pour une observation $X_i$ peut être définie comme

\begin{gather}
Z_i = \frac{|X_i-\mu|}{\sigma}.
\end{gather}

Cette valeur $Z$ nous donne en fait le nombre d'écarts-types dont une observation dévie de la moyenne. On peut généralement penser que si une observation dévie beaucoup de la moyenne, cela peut constituer un indicateur important d'anomalie. Dans ce cas-ci, on utilise généralement la règle du pouce $Z_i \ge 3$ comme indicateur d'anomalie. En d'autres mots, une observation est improbable, ou potentiellement "anormale", si elle se situe à plus ou moins 3 écarts-types de la moyenne. Ce modèle de détection d'anomalies s'applique bien dans le cas où nos observations sont indépendantes et proviennent réellement toutes d'une même loi normale. Dans la figure \ref{fig:Ztest_a}, on peut voir 5000 simulations d'une loi normale de moyenne nulle et d'écart-type égal à 2. On peut voir que notre règle du pouce, ou notre modèle, fonctionne relativement bien puisque les valeurs se situant à l'extérieur de l'intervalle $[-6, 6]$ ont tous une fréquence relative très faible. Cela nous laisse croire que la détection d'anomalies est adéquate. 

Dans une autre situation où les données ne proviennent pas d'une loi normale, notre règle du pouce, ou notre modèle, peut être moins adéquat. Supposons que nos données proviennent plutôt d'une loi à queue lourde, comme la loi standard Cauchy. En simulant $5000$ observations de cette loi, nous obtenons une moyenne empirique de $-1.6$ et un écart-type empirique de $160.2$. C'est donc dire que l'intervalle défini avec le modèle précédent serait de $[-482.2, 479]$. On peut voir à la figure \ref{fig:Ztest_b} que plusieurs observations ne sont pas considérées comme des anomalies selon ce modèle, alors que leur fréquence relative est pourtant très faible.

\begin{figure}[htb]
	\centering
	\begin{subfigure}{6cm}
		\centering\includegraphics[width=6cm]{images/histogram-normal-ztest}
		\caption{Distribution normale}
		\label{fig:Ztest_a}
	\end{subfigure}
	\begin{subfigure}{6cm}
		\centering\includegraphics[width=6cm]{images/histogram-cauchy-ztest}
		\caption{Distribution standard Cauchy}
		\label{fig:Ztest_b}
	\end{subfigure}
	\caption{Distribution de 5000 simulations d'une loi normale (a) et d'une loi standard Cauchy (b).}
	\label{fig:ZTest}
\end{figure}

L'exemple du test $Z$ illustre simplement le fait que le choix du modèle aura un impact important dans la détection d'anomalies. Au final, la pertinence du modèle choisi sera étroitement liée à la nature des données. Il existe plusieurs catégories d'algorithmes de détection d'anomalies. Dans \cite{10.5555/3086742}, quatre de ces approches nous apparaissent comme particulièrement intéressantes à survoler.

\subsection{Analyse des valeurs extrêmes}

La détection d'anomalies via l'analyse des valeurs extrêmes est probablement une des approches les plus simples. Dans un cas à une seule dimension, les anomalies sont définies comme les valeurs très grandes ou très faibles par rapport à la majorité des autres valeurs. On s'intéresse donc à la queue d'une distribution, comme dans l'exemple avec la statistique $Z$ présentée plus tôt. 

Bien que l'analyse des valeurs extrêmes ne soit pas la méthode la plus utilisée en pratique pour identifier des anomalies, elle fait souvent partie intégrante de plusieurs autres méthodes. En effet, l'analyse des valeurs extrêmes est souvent utilisée comme étape finale de plusieurs autres algorithmes. C'est d'ailleurs ce qui permet généralement de déterminer ce qui dévie d'une certaine normalité, via une valeur unidimensionnelle comme un score d'anomalie ou une distance.


\subsection{Les modèles probabilistes}

L'élément clé des approches basées sur des modèles probabilistes  est qu'on fait une hypothèse sur la distribution des données. Ensuite, on ajuste le modèle statistique correspondant à cette hypothèse sur les données en faisant l'apprentissage des paramètres du modèle. Un exemple classique pourrait être d'ajuster un mélange de $k$ lois normales. Avec ce modèle, on fait l'hypothèse que chaque observation provient d'un des $k$ groupes de la loi mélange dont une ou plusieurs de ces classes peuvent correspondre à des anomalies. On peut faire l'apprentissage des paramètres du modèle avec l'algorithme \textit{expectation-maximization (EM)}. Par la suite, on peut évaluer l'anormalité d'une observation en se basant sur le modèle ajusté. Ce type d'approche a l'avantage de pouvoir s'appliquer assez facilement à plusieurs types de données. Par contre, le désavantage est que l'on doit faire une hypothèse quant à la distribution des données, qui peut parfois être inadéquate. Certains jeux de données peuvent difficilement être représentés par une distribution connue, ce qui peut mener à un mauvais ajustement du modèle et ainsi tirer de mauvaises conclusions quant à la nature d'une observation.

\subsection{Les modèles linéaires et non-linéaires} \label{soussec:linear}

Ce type d'approche est basé sur l'apprentissage d'un espace à plus petites dimensions via un modèle linéaire ou non-linéaire. Un exemple classique est l'utilisation d'une régression linéaire. Par exemple, dans un cas d'une régression linéaire à 2 variables explicatives, l'apprentissage d'une variable réponse $y_i$ est donné par l'équation \ref{eq_reg}. Dans ce cas-ci, on utilise généralement la méthode des moindres carrés pour estimer les paramètres du modèle, soit les paramètres $\beta_0$, $\beta_1$ et $\beta_2$.

\begin{gather}  \label{eq_reg}
y_i = \beta_0 + \beta_{1} x_{1,i} + \beta_2 x_{2,i} + \epsilon_{i} \qquad \forall i \in {1,...,N}.
\end{gather}

Les estimations de ces paramètres, soit $\hat{\beta_0}$, $\hat{\beta_1}$ et $\hat{\beta_2}$, nous permettent d'estimer la variable réponse $\hat{y_i}$: 

\begin{gather}  \label{eq_reghat}
\hat{y_i} = \hat{\beta_0} + \hat{\beta_{1}} x_{1,i} + \hat{\beta_2} x_{2,i} \qquad \forall i \in {1,...,N}.
\end{gather}

On peut ensuite utiliser l'erreur de prédiction comme indicateur d'anomalie. Cette erreur de prédiction, définie comme $\hat{y_i} - y_i$, est également appelée erreur de reconstruction. La prémisse de base est que si le modèle est en mesure de bien s'ajuster aux données, les observations normales devraient être bien reconstruites par le modèle. À l'inverse, les observations anormales devraient être mal reconstruites par le modèle.  On peut d'ailleurs se servir encore une fois de l'analyse des cas extrêmes pour reconnaître les erreurs les plus importantes, et par le fait même, les anomalies.
 
Un autre exemple classique dans cette catégorie d'approches est l'utilisation d'une analyse en composante principale (ACP). Dans ce cas-ci, nous n'avons pas de variable réponse $y_i$ sur laquelle calculer une erreur de reconstruction. L'erreur de reconstruction sera donc basée sur la capacité de reconstruire l'entrée $x_i$. Cette technique de réduction de dimensionnalité permet de transformer les données vers un espace où les variables sont décorrélées et où l'utilisation d'un sous-ensemble de ces variables latentes peut être suffisant pour expliquer une bonne partie de la variabilité des données originales. La transformation permettant d'obtenir un espace latent à $k$ dimensions est donnée par l'équation \ref{eq_pca1}.
 
 \begin{gather}  \label{eq_pca1}
 Z = XV.
 \end{gather}
 
Dans l'équation \ref{eq_pca1}, $X$ est une matrice $n \times p$ représentant les $n$  observations initiales de dimensions $p$, mais où le vecteur de moyennes $\bar{X}$, calculé sur chaque colonne, a été soustrait à la matrice originale. $X$ est donc la matrice centrée des données initiales. $V$ représente une matrice $p \times k$, où $k < p$, de  vecteurs propres correspondant aux valeurs propres les plus élevées de la matrice de corrélation $\Sigma = n^{-1}X^\top X$. La valeur $k$ signifie qu'on conserve seulement $k$ variables latentes, ou également appelées composantes principales. La reconstruction des observations dans cet espace latent à $k$ dimensions peut ensuite être retrouvée par l'équation \ref{eq_pca2}:
 
  \begin{gather}  \label{eq_pca2}
 \hat{X} = ZV^\top + \bar{X}.
 \end{gather}
 
La projection de cet espace latent vers l'espace original défini par l'équation \ref{eq_pca2} permet d'obtenir une erreur de reconstruction. Celle-ci est peut être obtenue par une fonction appliquée entre $\hat{X}$ et $X$. Par exemple, on pourrait prendre la distance quadratique moyenne entre chaque instance de $\hat{X}$ et $X$ et ainsi obtenir $n$ erreurs de reconstruction. Cette erreur de reconstruction peut être utilisée pour trouver les observations qui dévient de la normalité du modèle de la même manière qu'avec une régression linéaire.
  
Ce type d'approche de détection d'anomalies a le potentiel de mieux s'adapter à des données où il n'y a pas de distribution connue. La régression linéaire est un exemple de modèle linéaire, mais on pourrait également utiliser un modèle plus complexe qui permet de capturer des non-linéarités. Par exemple, il est possible d'utiliser des réseaux de neurones pour faire l'apprentissage des données vers un espace à plus basses dimensions. Le désavantage avec de telles approches est qu'on peut perdre la notion d'interprétabilité. Dans certains exemples d'applications, il peut être intéressant de savoir quelles raisons expliquent la présence d'une anomalie dans les données.

\subsection{Les méthodes basées sur les distances}

Cette catégorie de méthodes de détection d'anomalies est basée sur l'idée de trouver les observations qui sont isolées de la majorité des autres observations. Cela est généralement quantifié par des mesures de distances ou de dissimilarités. Parmi ces méthodes, on peut retrouver 3 sous-catégories fréquemment utilisées en pratique: les regroupements (\textit{clustering}), les méthodes de densité et les méthodes des plus proches voisins.

Dans le cas des regroupements et des méthodes de densité, l'objectif est de trouver des zones de l'espace qui caractérisent le jeu de données. Les anomalies sont généralement identifiées en considérant les observations qui ne font pas parties de ces zones. Dans le cas spécifique des regroupements, on fait une séparation de l'espace qui est basée sur les observations elles-mêmes. Si on prend comme exemple l'algorithme de regroupement $k$-moyennes, on peut évaluer le score d'anomalie d'une observation en prenant sa distance minimale du centroïde des classes. Un autre exemple de méthode pouvant partie de cette catégorie sont les \textit{Isolation forest} \citep{4781136}. Dans cette approche, on tente d'isoler les anomalies par des regroupements hiérarchiques réalisés sur des échantillons \textit{bootstrap} des données. Nous donnerons davantage de détails sur cette approche dans la section \ref{isof_vae}.

Dans le cas des méthodes de densité, on partitionne l'espace des données en $k$ zones \textit{a priori}, ensuite on évalue la densité des observations dans chaque zone en fonction du nombre d'observations dans cette zone. Un exemple simple serait d'utiliser la méthode de l'histogramme afin de compartimenter l'espace en plusieurs sous-espaces et ensuite évaluer la densité d'une région par le nombre d'observations s'y retrouvant. Ce genre d'approche a généralement l'avantage d'être interprétable, car on peut savoir exactement pourquoi une observation fait partie d'un sous-ensemble ayant une faible densité. Prenons un exemple simple à seulement 2 dimensions où on retrouve le poids et la grandeur d'une personne. En séparant l'espace en différentes régions, on réalise que seulement très peu de personnes ont un poids inférieur à 45 kg et une grandeur supérieure à 1.80 m. Ainsi, on peut facilement expliquer pourquoi une personne de 39 kg et de 1.85 m est considérée comme une anomalie. 
 
 Finalement, les méthodes basées sur les plus proches voisins permettent de définir un score d'anomalie qui dépend de la distance entre une observation et ses $k$ plus proches voisins. Plus cette distance est grande, plus on peut penser que la donnée est isolée du reste des autres observations. Encore une fois, on peut avoir recours à l'analyse des cas extrêmes pour évaluer à partir de quelle distance on peut considérer une donnée comme une anomalie.


\section{Les autoencodeurs}

Maintenant que nous avons couvert certaines notions de base par rapport à différentes approches de détection d'anomalies, nous allons couvrir la théorie de base derrière le fonctionnement des autoencodeurs. Comme mentionné dans la section \ref{soussec:linear}, certaines approches de modélisation plus complexes, comme les autoencodeurs, peuvent être utilisées pour faire l'apprentissage d'un jeu de données. 

Les premiers travaux se rapprochant de l'autoencodeur connu aujourd'hui remontent aux années 1980 \citep{Rumelhart-1986}. Dans ces travaux, le principe était d'apprendre une représentation cachée en utilisant l'entrée $x_i$ elle-même comme variable réponse à prédire. Avec la montée en popularité des réseaux de neurones au début des années 2000, une nouvelle génération d'autoencodeurs ont vu le jour. Ces autoencodeurs avaient désormais plusieurs couches cachées superposées \citep{HintonSalakhutdinov2006b}, permettant ainsi de faire l'apprentissage de données plus complexes comme des images. Un autoencodeur est un réseau de neurones qui a comme objectif d'apprendre une représentation intermédiaire d'une entrée de manière non-supervisée \citep{Goodfellow-et-al-2016}. Pour réaliser cet objectif, l'autoencodeur se décompose en 2 composantes: un encodeur et un décodeur. L'encodeur reçoit en entrée $x$ et convertit celle-ci vers une représentation latente $z$. Le décodeur prend en entrée cette représentation latente $z$ et la décode pour ainsi retrouver une sortie qui doit se rapprocher le plus possible de l'entrée initiale $x$. Cette structure de base est illustrée dans la figure \ref{fig:basicAE}. Historiquement, les autoencodeurs étaient vus comme une méthode de réduction de dimensionnalité, mais désormais, ceux-ci ont davantage d'applications liées au fait qu'ils peuvent apprendre des variables latentes riches en informations. On peut citer des applications en vision numérique permettant de retrouver le contenu d'une image \citep{conf/esann/KrizhevskyH11} ou même en traitement de la langue naturelle pour interpréter un discours oral \citep{inproceedings}. \newline

\begin{figure}[h]
	\centering
	\begin{tikzpicture}[shorten >=1pt,draw=black!50, node distance=\layersep, square/.style={regular polygon,regular polygon sides=4}]
	\tikzstyle{every pin edge}=[<-,shorten <=1pt]
	\tikzstyle{neuron}=[square,fill=black!25,minimum size=17pt,inner sep=0pt]
	\tikzstyle{input neuron}=[neuron, fill=green!50];
	\tikzstyle{output neuron}=[neuron, fill=red!50];
	\tikzstyle{hidden neuron1}=[neuron, fill=blue!50];
	\tikzstyle{hidden neuron2}=[neuron, fill=blue!50];
	\tikzstyle{hidden rep}=[neuron, fill=yellow!50];
	\tikzstyle{annot} = [text width=4em, text centered]
	
	% Draw the input layer nodes
	\foreach \name / \y in {1,...,4}
	% This is the same as writing \foreach \name / \y in {1/1,2/2,3/3,4/4}
	\node[input neuron] (I-\name) at (0,-\y) {};
	
	% Draw the hidden layer nodes n.1
	\foreach \name / \y in {1,...,2}
	\path[yshift=-1cm]
	node[hidden neuron1] (H1-\name) at (\layersep,-\y cm) {};
	
	% Draw the encoded representation
	\foreach \name / \y in {1,...,1}
	\path[yshift=-1.5cm]
	node[hidden rep] (R-\name) at (2 * \layersep,-\y cm) {};
	
	% Draw the hidden layer nodes n.2
	\foreach \name / \y in {1,...,2}
	\path[yshift=-1cm]
	node[hidden neuron1] (H2-\name) at (3 * \layersep,-\y cm) {};
	
	% Draw the output layer
	\foreach \name / \y in {1,...,4}
	% This is the same as writing \foreach \name / \y in {1/1,2/2,3/3,4/4}
	\node[output neuron] (O-\name) at (4 * \layersep,-\y cm) {};
	
	% Connect input
	\foreach \source in {1,...,4}
	\foreach \dest in {1,...,2}
	\path (I-\source) edge (H1-\dest);
	
	% Connect representation
	\foreach \source in {1,...,2}
	\foreach \dest in {1,...,1}
	\path (H1-\source) edge (R-\dest);
	
	\foreach \source in {1,...,1}
	\foreach \dest in {1,...,2}
	\path (R-\source) edge (H2-\dest);
	
	% Connect outputs
	\foreach \source in {1,...,2}
	\foreach \dest in {1,...,4}
	\path (H2-\source) edge (O-\dest);
	
	% Annotate the layers
	\node[annot,above of=I-1, node distance=1cm][text width=6em] (hl) {Couche d'entrée ($x$)};
	\node[annot,above of=R-1, node distance=2.5cm][text width=8em] (hl) {Représentation \\ latente ($z$)};
	\node[annot,above of=O-1, node distance=1cm][text width=6em] (hl) {Couche de sortie ($\hat{x}$)};
	
	\draw [decorate,decoration={brace,mirror,amplitude=15pt},xshift=-4pt,yshift=-2cm]
	(0,-2.5) -- (4,-2.5) node [black,midway,yshift=-3em] 
	{\footnotesize encodeur: $q_{\theta}(x)$};
	\draw [decorate,decoration={brace,mirror,amplitude=15pt},xshift=4pt,yshift=-2cm]
	(4,-2.5) -- (8,-2.5) node [black,midway,yshift=-3em] 
	{\footnotesize décodeur: $p_{\phi}(z)$};
	
	\end{tikzpicture}
	\caption{Exemple illustrant la structure de base d'un autoencodeur. Dans le cas ci-dessus, on pourrait interpréter le schéma comme un réseau pleinement connecté où les blocs pourraient représenter les neurones et les liens seraient les poids, ou les paramètres du réseau.}
	\label{fig:basicAE}
\end{figure}

L'intuition derrière les autoencodeurs est de reconstruire une entrée $x$ en passant par 2  fonctions (l'encodeur et le décodeur) apprises par le modèle. Ces deux fonctions vont permettre d'obtenir une sortie $\hat{x}$, donnée par l'équation:

\begin{gather} \label{eq:output}
	\hat{x} = p_\phi{\{q_\theta(x)\}}.
\end{gather}

 Ainsi, ce genre de méthode n'a pas besoin d'une étiquette $y$, car son objectif est de reproduire la valeur de $x$ elle-même; c'est d'ailleurs pour cela qu'on parle d'une approche d'apprentissage non-supervisée. L'apprentissage des paramètres est fait en minimisant l'erreur de reconstruction. La perte peut donc être définie par une fonction de la forme

\begin{gather}  \label{eq:loss}
L(x, \hat{x}) = L(x, p_\phi{\{q_\theta(x)\}}),
\end{gather}


\noindent où $q_{\theta}(\cdot)$ est l'encodeur et $p_{\phi}(\cdot)$ est le décodeur. L'optimisation de cette fonction de perte est faite par descente du gradient. En d'autres mots, les paramètres $\theta$ de l'encodeur et $\phi$ du décodeur sont optimisés graduellement en prenant la dérivée de la fonction de perte par rapport aux différents paramètres de ces deux composantes:


\begin{equation} \label{optim}
\Theta^{'} \leftarrow \Theta-\epsilon*\frac{\partial L}{\partial\Theta},
\end{equation}

\noindent où $\epsilon$ est un taux d'apprentissage qui permet de moduler la vitesse d'apprentissage et $\Theta = \{\theta, \phi\}$ comprend les paramètres de l'encodeur et du décodeur. $\Theta^{'}$; correspond aux valeurs des paramètres à la suite d'une itération. Cette approche d'optimisation s'appelle la rétropropagation du gradient. Il s'agit d'une composante clé dans l'optimisation des différentes couches des réseaux de neurones.

Pour mieux comprendre l'entraînement d'un autoencodeur, illustrons l'apprentissage d'un autoencodeur de base avec un exemple simpliste. Supposons que nous avons en entrée un jeu de données $X$ avec $p=2$ variables et $n=10$ observations. Nous voulons encoder ces 2 variables dans une variable latente unidimensionnelle avec un autoencodeur à une seule couche cachée. Au total, notre autoencodeur possède 3 couches, soit une couche d'entrée, une couche cachée et une couche de sortie. Nous choisissons également une fonction d'activation sigmoïde pour notre couche cachée et une fonction d'activation linéaire pour notre couche de sortie. Les fonctions d'activation sont des transformations appliquées aux valeurs des neurones d'une couche du réseau, ce qui permettra entre autres d'avoir un modèle non-linéaire, dans le cas de fonctions d'activation non-linéaires. L'architecture du réseau est illustrée dans la figure \ref{fig:toyAE}.

\begin{figure}[htb]
	\centering
	\begin{tikzpicture}
	[   cnode/.style={draw=black,fill=#1,minimum width=3mm,circle},
	]
	\node[cnode=blue,label=180:$x_{1}$] (x-1) at (0,2) {};
	\node[cnode=blue,label=180:$x_{2}$] (x-2) at (0,-2) {};
	\node[cnode=gray,label=90:$z_{1}$] (z) at (3,0) {};
	\node[cnode=red,label=0:$\hat{x}_{1}$] (x_chap-1) at (6,2) {};
	\node[cnode=red,label=0:$\hat{x}_{2}$] (x_chap-2) at (6,-2) {};
	
	\foreach \x in {1,2}
	{   \draw (x-\x) -- node[above,sloped]{$\omega_{1,\x}$} (z);
		\draw (x_chap-\x) -- node[above,sloped]{$\omega_{2,\x}$} (z);
	}
	
	\node[draw=white] at (1.5,-1.7) {$b_{1}$};
	\node[draw=white] at (4.5,-1.7) {$b_{2}$};
	
	\draw [decorate,decoration={brace,mirror,amplitude=10pt},xshift=-4pt,yshift=0pt]
	(0,-2.5) -- (3,-2.5) node [black,midway,yshift=-2em] 
	{\footnotesize encodeur: $q_{\theta}(x)$};
	\draw [decorate,decoration={brace,mirror,amplitude=10pt},xshift=4pt,yshift=0pt]
	(3,-2.5) -- (6,-2.5) node [black,midway,yshift=-2em] 
	{\footnotesize décodeur: $p_{\phi}(z)$};
	
	\end{tikzpicture}
	\caption{Exemple simpliste d'une architecture d'autoencodeur}
	\label{fig:toyAE}
\end{figure}

La fonction correspondant à l'encodeur, qui transforme l'entrée $x$ vers la représentation latente $z$, est définie par l'équation \ref{eq:encodeur1}. Dans cette équation, $h(\cdot)$ est la fonction d'activation sigmoïde donnée par : $h(x)=\frac{1}{1+e^{-x}}$. Les paramètres $\theta$ qui doivent être optimisés sont $w_{1,1}, w_{1,2}$ et $b_1$. 

\begin{gather}  \label{eq:encodeur1}
q_{\theta}(x)=h\big(x_1w_{1,1} + x_2 w_{1,2} + b_1\big).
\end{gather}

Pour définir l'encodeur, il est également possible d'utiliser la notation matricielle définie par l'équation \ref{eq:encodeurmat} où $X = [\boldsymbol x_1, \boldsymbol x_2]$. Pour une seule observation, on peut définir $X^{(i)} = [x_{1}^{(i)}, x_{2}^{(i)}]$. Le vecteur de poids $\boldsymbol w_1$ est défini comme $\boldsymbol w_1 = [w_{1,1}, w_{1,2}]$:

\begin{gather}  \label{eq:encodeurmat}
q_{\theta}(X)=h\big(X*\boldsymbol w_1 + b_1\big).
\end{gather}

De la même manière que pour l'encodeur, il est possible de définir de manière plus détaillée la fonction associée au décodeur (équation \ref{eq:decodeur}).

\begin{gather}  \label{eq:decodeur}
p_{\phi}(Z)=
\begin{cases}
p_{\phi}^1(z) = z*w_{2,1}+b_2, \\
p_{\phi}^2(z) = z*w_{2,2}+b_2.
\end{cases}
\end{gather}

Les paramètres $\phi$ qui doivent être optimisés sont $w_{2,1}, w_{2,2}$ et $b_2$. Encore une fois, il est possible d'utiliser une notation matricielle, comme définie à l'équation \ref{eq:decodeurmat} où $\boldsymbol w_2 = [w_{2,1}, w_{2,2}]$:

\begin{gather} \label{eq:decodeurmat}
p_{\phi}(Z)=h\big(Z*\boldsymbol w_2 + b_2\big)=\hat{X}, \qquad
p_{\phi}: \mathbb{R}\rightarrow \mathbb{R}^2.
\end{gather}

Pour trouver les valeurs optimales des paramètres $\theta$ et $\phi$ du modèle, nous devons définir une fonction de perte (équation \ref{eq:loss}). Supposons que nous voulons minimiser l'erreur de reconstruction seulement, nous pouvons alors définir notre fonction de perte de la manière suivante:

\begin{equation} \label{perte1}
\begin{split}
L(X,\hat{X}) & = L\big(X, p_\phi\{q_\theta(X)\}\big) \\
& = \frac{1}{2} \sum_{i=1}^{n} [x^{(i)}-p_\phi\{q_\theta(x^{(i)})\}]^{\text{T}}[x^{(i)}-p_\phi\{q_\theta(x^{(i)})\}] \\
& = \frac{1}{2} \sum_{i=1}^{n} \sum_{j=1}^{p} \big(x_{j}^{(i)}-p^j_\phi\{q_\theta(x_{j}^{(i)})\}\big)^2,
\end{split}
\end{equation}

où $i$ représente la $i$-ième observation et $j$ représente le $j$-ième élément de l'entrée $X$. Maintenant que nous avons notre fonction de perte, nous pouvons optimiser les paramètres de nos fonctions encodeur et décodeur en utilisant la règle de dérivation en chaîne sur chacune des couches de notre réseau (voir équation \ref{optim}).

Une fois que l'autoencodeur est entraîné, il est possible d'utiliser l'erreur de reconstruction comme score d'anomalie. En effet, une donnée mal reconstruite pourrait être vue comme une donnée aberrante, ou une donnée que le réseau n'a pas eu l'habitude de voir lors de l'entraînement. Dans  \cite{10.5555/3086742}, ce genre de méthode de détection d'anomalies fait partie de la catégorie des algorithmes basés sur les modèles linéaires ou non-linéaires. Dans ce genre d'approche, on fait généralement l'apprentissage du modèle par l'erreur de reconstruction, où l'on cherche à minimiser les erreurs de prédiction. Toutefois, l'erreur de reconstruction n'est pas le seul critère qui peut être utilisé pour entraîner un autoencodeur. Dans la prochaine section, nous allons considérer un type précis d'autoencodeur, soit l'autoencodeur variationnel. Nous verrons également quelle autre composante peut être utilisée comme score d'anomalie.

\section{Les autoencodeurs variationnels} \label{background-vae}

Les autoencodeurs variationnels (VAE)  \cite{kingma2013autoencoding} ont une approche légèrement différente des autres autoencodeurs. En effet, au lieu d'encoder les données dans un vecteur de variables latentes à $p$ dimensions, les données sont plutôt encodées dans 2 vecteurs de taille $p$: un vecteur de moyennes $\boldsymbol \mu$ et un vecteur d'écarts-types $\boldsymbol \sigma$. Ces deux vecteurs sont ensuite utilisés comme paramètres d'une distribution paramétrique utilisée pour générer aléatoirement une représentation latente à $p$ dimensions. Cette dernière représentation latente correspond donc dans ce cas-ci à une simulation d'une loi normale multivariée avec les paramètres $\boldsymbol \mu$ et $\boldsymbol \sigma$. Cela permet donc d'obtenir pour une donnée $x$, une représentation latente continue. C'est d'ailleurs la particularité la plus importante des VAE par rapport aux autres autencodeurs. En d'autres mots, les autoencodeurs de base ont comme objectif d'apprendre une représentation latente discrète, alors que les VAE apprennent une représentation continue qui représente plutôt une zone de l'espace latent. Cette zone est effectivement définie par les paramètres $\boldsymbol \mu$ et $\boldsymbol \sigma$ associés à une entrée donnée. Cela veut aussi dire qu'une fois l'algorithme entraîné, la sortie de celui-ci est stochastique dans le sens où deux encodages d'une même entrée $x$ peuvent donner deux sorties différentes. Cependant, une donnée $x$ va toujours mener aux mêmes valeurs de $\boldsymbol \mu$ et $\boldsymbol \sigma$, cette composante n'est donc pas stochastique. La figure \ref{fig:VAEstructure} illustre la structure de base des autoencodeurs variationnels. Dans la figure, on peut remarquer que les données en entrée commencent par passer par des couches cachées, qui peuvent être pleinement connectées ou de convolution. Une couche pleinement connectée est une couche où chaque neurone est connecté avec tous les neurones de la couche suivante. Cette connexion est faite via un produit matriciel entre les valeurs des neurones et les poids du modèle. Dans le cas d'une couche de convolution, le produit matriciel est obtenu entre les valeurs d'une matrice se déplaçant sur toutes les zones de la couche précédente. Cette matrice, également appelée un filtre, est généralement carrée et de petites dimensions. Les paramètres d'une couche de convolution sont les valeurs de cette matrice, ou de ce filtre. Contrairement aux couches pleinement connectées, les couches de convolution ont l'avantage d'avoir beaucoup moins de paramètres, ce qui devient un aspect crucial en imagerie numérique où les données d'entrées sont généralement très complexes. De plus, les filtres se déplaçant sur l'image permettent de traiter l'information de manière localisée, ce qui est également un aspect important dans le traitement des images.

 Comme les autoencodeurs traditionnels, la première partie de l'autoencodeur variationnel est l'encodeur ($q_{\theta}(\cdot)$). Un peu avant d'arriver à la représentation latente, le réseau se divise en 2 composantes ($\boldsymbol \mu$ et $\boldsymbol \sigma$). La représentation latente $z$ est ensuite générée en simulant d'une loi normale multivariée avec les paramètres $\boldsymbol \mu$ et $\boldsymbol \sigma$. C'est la fonction qu'on désigne par $q_{\theta}(z|x)$ dans la figure \ref{fig:VAEstructure}. Une fois la représentation $z$ générée, celle-ci est décodée jusqu'au format original par des couches pleinement connectées ou des couches de déconvolution. Pour les couches de déconvolution, on fait l'opération inverse de la convolution. Cette partie est le décodeur ($p_{\phi}(z)$).

\begin{figure}[h]
	\centering
	\begin{tikzpicture}[shorten >=1pt,draw=black!50, node distance=\layersep, square/.style={regular polygon,regular polygon sides=4}]
	\tikzstyle{every pin edge}=[<-,shorten <=1pt]
	\tikzstyle{neuron}=[square,fill=black!25,minimum size=17pt,inner sep=0pt]
	\tikzstyle{input neuron}=[neuron, fill=blue!50];
	\tikzstyle{output neuron}=[neuron, fill=blue!50];
	\tikzstyle{hidden neuron1}=[neuron, fill=blue!50];
	\tikzstyle{hidden neuron2}=[neuron, fill=blue!50];
	\tikzstyle{sample rep}=[neuron, fill=yellow!50];
	\tikzstyle{hidden rep}=[neuron, fill=red!50];
	\tikzstyle{annot} = [text width=4em, text centered]
	
	% Draw the input layer nodes
	\foreach \name / \y in {1,...,4}
	% This is the same as writing \foreach \name / \y in {1/1,2/2,3/3,4/4}
	\node[input neuron] (I-\name) at (0,-\y) {};
	
	% Draw the hidden layer nodes n.1
	\foreach \name / \y in {1,...,3}
	\path[yshift=-0.5cm]
	node[hidden neuron1] (H1-\name) at (\layersep,-\y cm) {};
	
	% Draw the mu and sigma layers
	\foreach \name / \y in {1,...,2}
	\path[yshift=-1cm]
	node[hidden rep] (R-\name) at (2 * \layersep,-\y cm) {};
	
	% Draw the encoded representation
	\foreach \name / \y in {1,...,1}
	\path[yshift=-1.5cm]
	node[sample rep] (S-\name) at (3 * \layersep,-\y cm) {};
	
	% Draw the hidden layer nodes n.2
	\foreach \name / \y in {1,...,3}
	\path[yshift=-0.5cm]
	node[hidden neuron1] (H2-\name) at (4 * \layersep,-\y cm) {};
	
	% Draw the output layer
	\foreach \name / \y in {1,...,4}
	% This is the same as writing \foreach \name / \y in {1/1,2/2,3/3,4/4}
	\node[output neuron] (O-\name) at (5 * \layersep,-\y cm) {};
	
	% Connect input
	\foreach \source in {1,...,4}
	\foreach \dest in {1,...,3}
	\path (I-\source) edge (H1-\dest);
	
	% Connect hidden 1 with mu and sigma
	\foreach \source in {1,...,3}
	\foreach \dest in {1,...,2}
	\path (H1-\source) edge (R-\dest);
	
	% Connect representation
	\foreach \source in {1,...,2}
	\foreach \dest in {1,...,1}
	\path (R-\source) edge (S-\dest);
	
	\foreach \source in {1,...,1}
	\foreach \dest in {1,...,3}
	\path (S-\source) edge (H2-\dest);
	
	% Connect outputs
	\foreach \source in {1,...,3}
	\foreach \dest in {1,...,4}
	\path (H2-\source) edge (O-\dest);
	
	% Annotate the layers
	\node[annot,above of=I-1, node distance=1cm] (hl) {Couche d'entrée};
	\node[annot,above of=S-1, node distance=0cm][text width=8em] (hl) {$\boldsymbol z$};
	\node[annot,above of=S-1, node distance=1cm][text width=8em] (hl) {$q_{\theta}(z|x) \sim N(\boldsymbol \mu, \boldsymbol \sigma)$};
	\node[annot,above of=R-1, node distance=0cm][text width=8em] (hl) {$\boldsymbol \mu$};
	\node[annot,above of=R-2, node distance=0cm][text width=8em] (hl) {$\boldsymbol \sigma $};
	\node[annot,above of=O-1, node distance=1cm] (hl) {Couche de sortie};
	
	\draw [decorate,decoration={brace,mirror,amplitude=15pt},xshift=-4pt,yshift=-2cm]
	(0,-2.5) -- (6,-2.5) node [black,midway,yshift=-3em] 
	{\footnotesize encodeur: $q_{\theta}(x)$};
	\draw [decorate,decoration={brace,mirror,amplitude=15pt},xshift=4pt,yshift=-2cm]
	(6,-2.5) -- (10,-2.5) node [black,midway,yshift=-3em] 
	{\footnotesize décodeur: $p_{\phi}(z)$};
	
	\end{tikzpicture}
	\caption{Structure de base d'un autoencodeur variationnel. La représentation latente $z$ est générée en simulant de la loi $q_{\theta}(z|x)$,  qui suit une loi normale multivariée avec les paramètres $\boldmath \mu$ et $\boldmath \sigma$ calculés par le réseau. C'est de là que vient la composante stochastique du modèle.}
	\label{fig:VAEstructure}
\end{figure}

Les autoencodeurs variationnels sont également différents quant au calcul de perte qui est essentiel à l'optimisation du réseau. En effet, on ajoute une autre composante de perte à l'erreur de reconstruction de base. La fonction de perte est désormais définie comme une somme de deux composantes:

\begin{gather}  \label{eq:loss_vae}
L(x, p_\phi\{q_\theta(x)\}) + D_{KL}\big[q_\theta(z|x) || p(z)\big],
\end{gather}


où $D_{KL}$ est une divergence de Kullback-Leibler. Cette mesure statistique permet de quantifier la différence entre deux distributions. Pour deux distributions continues $A$ et $B$, la divergence de Kullback-Leibler est donnée par

\begin{equation}  \label{eq:kl}
	\begin{aligned}
		D_{\text{KL}}(A || B) &= \int_{-\infty}^{\infty} a(x) \text{log} \Big(\frac{a(x)}{b(x)}\Big)  \\
		 &= \int_{-\infty}^{\infty} a(x) (\text{log}(a(z)) - \text{log}(b(z))).
	\end{aligned}
\end{equation}

Dans le cas précis de l'autoencodeur variationnel, la distribution $q_{\theta}(z|x)$ est une loi normale multivariée de paramètres $\boldsymbol \mu$ et $\boldsymbol \sigma$. La loi $p(z)$, où la loi \textit{a priori}, est une loi normale multivariée standard $N(0,I)$, où $I$ est la matrice identitaire. Dans le calcul de perte, plus la simulation provenant de la distribution $q_{\theta}(z|x)$ diverge d'une loi $N(0, I)$, plus la perte sera importante. Cela permet de régulariser la représentation latente dans un espace défini par la distribution \textit{a priori}. Pour le calcul du critère de Kullback-Leibler entre une loi $N(\boldsymbol \mu, \boldsymbol \sigma)$ et une loi $N(0, I)$ à $k$ dimensions, l'équation \ref{eq:kl} se développe comme suit:

\begin{equation}  \label{eq:kl2}
	\begin{aligned}
	D_{\text{KL}}(N_{\boldsymbol \mu, \boldsymbol \sigma} || N_{0,I}) &= \int_{-\infty}^{\infty} N_{\boldsymbol \mu, \boldsymbol \sigma}(x) \big\{\text{log}(N_{\boldsymbol \mu,\boldsymbol \sigma}(x)) - \text{log}(N_{0,I}(x))\big\} \\
		&= \frac{1}{2}\big\{\text{tr}(\boldsymbol{\sigma^2}) + \boldsymbol{\mu^T} \boldsymbol{\mu} - k - \text{log det}(\boldsymbol{\sigma^2})\big \}.
 	\end{aligned}
\end{equation}.

La trace calculée sur $\boldsymbol{\sigma^2}$ est la somme des variances sur les $k$ dimensions. Étant donné que la matrice de variance-covariance est supposée diagonale, comme la fonction \textit{a priori}, son déterminant se calcule comme le produit de sa diagonale. On peut donc simplifier l'équation \ref{eq:kl2} comme suit:

\begin{equation}  \label{eq:kl3}
\begin{aligned}
D_{\text{KL}}(N_{\boldsymbol \mu, \boldsymbol \sigma} || N_{0,I}) &=  \frac{1}{2}\Big\{\sum_k \sigma_{k}^2+ \sum_{k}\mu_{k}^2 - \sum_{k} 1 - \text{log} \prod_{k}\sigma_{k}^2\Big \} \\
&=\frac{1}{2}\Big\{\sum_k \sigma_{k}^2+ \sum_{k}\mu_{k}^2 - \sum_{k} 1 - \sum_{k} \text{log} (\sigma_{k}^2)\Big \} \\
&=\frac{1}{2} \sum_{k}\Big\{\sigma_{k}^2 + \mu_{k}^2 - 1 - \text{log}(\sigma^2)\Big\}.
\end{aligned}
\end{equation}

Pour des raisons de stabilité numérique, on utilise souvent en pratique la version logarithme de la matrice $\boldsymbol{\sigma^2}$ (équation \ref{eq:kl3}). Par contre, il est aussi possible de réécrire l'équation \ref{eq:kl3} sous cette forme:

\begin{equation}  \label{eq:kl4}
\begin{aligned}
D_{\text{KL}}(N_{\boldsymbol \mu, \boldsymbol \sigma} || N_{0,I}) &= \frac{1}{2} \sum_{k}\Big\{\text{exp}(\sigma_{k}^2) + \mu_{k}^2 - 1 - \sigma^2\Big\}.
\end{aligned}
\end{equation}


 L'objectif de cette nouvelle composante est de s'assurer que la distribution encodée $q_{\theta}(z|x)$ et la distribution \textit{a priori} $p(z)$ sont similaires. Si cette composante de perte est suffisamment prise en compte lors de l'optimisation, nous devrions donc obtenir des paramètres $\boldmath \mu$ et $\boldmath \sigma$ se rapprochant d'un vecteur de 0 et d'une matrice identitaire. 

Comme expliquée un peu plus tôt, la représentation latente de ce type d'autoencodeur est simulée d'une loi normale multivariée, donc stochastique. Cela pourrait en théorie rendre complexe la tâche d'optimisation du réseau. En effet, les paramètres du réseau, soit ceux de l'encodeur et du décodeur, sont optimisés par descente du gradient. On calcule donc la perte $L(x)$ et on dérive cette fonction par rapport à chaque paramètre du réseau. Mais qu'en est-il de la loi normale multivariée qui a généré notre représentation latente? Afin de simplifier le calcul des dérivées lors de la rétropropagation, on redéfinit le calcul de la représentation $z$ dans un format plus simple. Cette simplification est communément appelée la  \textit{reparametrization trick}. En effet, il est possible pour certaines distributions, comme la loi normale, de séparer les paramètres ($\mu$ et $\sigma$)  de la composante stochastique. Concrètement, on peut définir une simulation normale multivariée comme:

\begin{equation} \label{eq:latent_formula}
\boldsymbol{z} = \boldsymbol{\mu} + \boldsymbol{\sigma} *\boldsymbol{\epsilon}
\end{equation}

où $\boldsymbol{\epsilon} \sim N(0,I)$. En bref, cela signifie que la couche $z$, illustrée dans la figure \ref{fig:VAEstructure}, est générée à partir de deux couches de paramètres $\boldsymbol{\mu}$ et $\boldsymbol{\sigma}$ et d'une simulation $N(0,I)$. Cette réécriture permet d'isoler la composante stochastique associée à la loi normale. Lors de la rétropropagation, on peut calculer les dérivées des paramètres $\boldsymbol{\mu}$ et $\boldsymbol{\sigma}$ seulement et ignorer $\boldsymbol{\epsilon}$. 

\subsection{$\beta$-VAE} \label{beta-vae}

La composante de perte associée au critère de divergence de Kullback-Leibler apporte de la régularisation au modèle en restreignant celui-ci dans un certain espace \citep{kingma2013autoencoding}. Cette régularisation est d'autant plus importante dans un contexte où les réseaux de neurones ont un potentiel d'apprentissage important et où il devient primordial d'éviter le sur-apprentissage. Par contre, trop de régularisation pourrait amener un réseau à sous-apprendre. Il est donc important de trouver le bon équilibre entre la perte associée à la reconstruction et la perte associée au critère de Kullback-Leibler (KL).

Pour trouver ce bon équilibre, on peut ajouter un hyperparamètre à notre fonction de perte. Cet hyperparamètre, défini par $\beta$, permet de donner plus ou moins d'importance au critère de KL. Avec cet hyperparamètre, la fonction de perte est maintenant donnée par

\begin{gather}  \label{eq:loss_betavae}
L(x, p_\phi\{q_\theta(x)\}) +  \beta \times D_{KL}\big[q_\theta(z|x) || p(z)\big].
\end{gather}

Dans ce cas-ci, on parle plutôt d'un $\beta$-VAE, qui a été introduit par \cite{Higgins2017betaVAELB}. Plus cet hyperparamètre $\beta$ est élevé, plus la régularisation est importante et aussi plus les éléments de la représentation latente seront dissociés les uns des autres (ce qu'on appelle une représentation \textit{disentangled} en anglais). Cela est dû à la covariance nulle de la loi normale multivariée utilisée comme fonction \textit{a priori} de $p(z)$, sur laquelle ce critère de perte est basé. Cette propriété de \textit{disentanglement} peut devenir intéressante dans le cas où on souhaite avoir des variables latentes indépendantes qui expliquent différents aspects des données. Un exemple cité dans leur article fait référence à un jeu de données synthétiques de visages. Après avoir entraîné un réseau qui permet d'obtenir des variables latentes \textit{disentangled}, les auteurs sont capables de démontrer que chaque variable latente a une fonction précise dans l'image reconstruite: rotation, éclairage, élévation, etc. Ils ont d'ailleurs illustré ces différentes fonctions en faisant varier les valeurs d'une variable latente à la fois. Ces variations permettaient de modifier différents aspects de l'image reconstruite, comme par exemple la rotation de l'image.

\subsection{La détection d'anomalies avec un VAE}

Pour ce qui est de la détection d'anomalies, les autoencodeurs variationnels, ou les $\beta$-VAE peuvent en théorie être utilisés de la même manière qu'un autoencodeur de base. En effet, le VAE calcule une erreur de reconstruction qui peut ensuite être utilisée comme score d'anomalie. Par contre, une autre composante du VAE peut potentiellement être intéressante dans un contexte de détection d'anomalie. C'est d'ailleurs autour de cela que tourne le sujet de ce mémoire. Il s'agit de la représentation latente qui est basée sur une distribution \textit{a priori}. Étant donné que la fonction de perte du VAE pénalise une représentation latente s'éloignant de cette distribution \textit{a priori}, il intéressant de voir comment se comporte cette représentation latente pour des observations "normales" comparativement à des observations "anormales". Cette représentation, potentiellement riche en informations, possède généralement un niveau de complexité bien inférieur à l'entrée. En bref, le fait d'obtenir une représentation simple et riche en informations sur laquelle on a un \textit{a priori} pourrait devenir une possibilité intéressante quant à la détection d'anomalies. C'est d'ailleurs un sujet que nous reverrons plus loin dans la description de la méthodologie.

Avec ce survol des méthodes de détection d'anomalies et des autoencodeurs, nous sommes en mesure de décrire plus en détails notre approche proposée dans le prochain chapitre.
             % background
\chapter{Méthodologie}     % numéroté
\label{chap:methodologie}                   % étiquette pour renvois (à compléter!)

Le coeur de notre travail consiste à proposer une méthode de détection d'anomalies où nous utilisons la représentation latente d'un autoencodeur variationnel qui sera transformée vers une métrique que nous pourrons tester via un test d'hypothèse. Ce test d'hypothèse nous permet donc de détecter des anomalies avec un niveau de confiance.

\section{Les hypothèses de l'approche}

Dans notre approche, nous supposons que  nous avons accès à un jeu de données qui contient presque qu'entièrement des observations normales. En d'autres mots, il n'y pratiquement pas d'anomalies. Par contre, nous anticipons des entrées anormales dans le futur et nous voulons les détecter. Voici 2 exemples de situation où ce genre de contexte pourrait être plausible:

\begin{enumerate}
	\item Une compagnie d'assurance propose maintenant à leurs assurés de prendre eux-mêmes des photos de leur véhicule accidenté et d'envoyer celles-ci à l'assureur. La compagnie s'attend à ce que les assurés puissent commettre des erreurs de manipulation en prenant les photos ou bien puissent envoyer les mauvaises photos. De son côté, l'assureur possède des photos de véhicules accidentés, prises par son personnel d'estimateurs, mais aucune photo erronée. Le jeu de données d'entraînement ne possède théoriquement aucune anomalie.
	\item Une entreprise manufacturière qui produit des pièces de voiture vient de se procurer un nouveau système qui automatise davantage la production. L'entreprise veut faire un suivi des pièces produites par ce nouveau système via une caméra qui inspecte les produits finaux. L'entreprise possède déjà plusieurs images de pièces qui étaient produites avant l'acquisition du nouveau système. Le résultat final produit par ce nouveau système doit être identique à l'ancien. Par contre, il est difficile de prévoir de quoi auront l'air des défectuosités produites par ce nouveau système. Le jeu de données d'entraînement est donc constitué de plusieurs exemples normales, mais aucune anomalie.
\end{enumerate}

\section{Description de l'approche}

Considérant que nous avons accès à un jeu de données ayant les caractéristiques décrites à la section précédente, nous proposons d'entraîner un autoencoder variationnel pour apprendre les caractéristiques, ou la distribution, de cette population "normale". Cette distribution apprise sera essentiellement contenu dans la représentation latente du VAE entraîné, ou plus spécifiquement dans les couches $\mu$ et $\sigma$ du réseau. Comme nous l'avons vu dans la section \ref{background-vae}, les VAE ont la particularité d'avoir une composante de perte associée à cette représentation latente, s'assurant ainsi que celle-ci s'approche d'une distribution à priori, soit une $N(0, I)$ dans notre cas. Une fois que l'autoencodeur est entraîné, nous pouvons utiliser la partie encodeur du réseau pour transformer chacune des instances de notre jeu de données d'entraînement vers des  vecteurs $\mu$ et $\sigma$. Ces représentations latentes, encodées sous forme de vecteurs, contiennent l'information qui devrait permettre de savoir si une nouvelle instance partage ou non cette même information. Pour parvenir à faire cette conclusion sur une nouvelle instance, nous devons comparer l'information contenue de la représentation latente de cette instances avec toutes les représentations latentes de l'ensemble d'entraînement. Il est donc plus pratique d'agréger notre population d'entraînement en une seule représentation globale. Une première manière d'y arriver est de calculer un vecteur $\bar{\mu}$ et un vecteur $\bar{\sigma}$, qui sont simplement une moyenne sur la population d'entraînement. Supposons que notre jeu de données d'entraînement contient $n$ observations et que nous encodons celles-ci vers des représentations latentes à $k$ dimensions, l'élément $i$ des vecteurs $\bar{\mu}$ et $\bar{\sigma}$ sera donné par :

\begin{gather*}
\bar{\mu}^{(i)} = \frac{1}{n} \sum_{k=1}^{n} \mu^{(k)} \\
\bar{\sigma}^{(i)} = \frac{1}{n} \sum_{k=1}^{n} \sigma^{(k)}
\end{gather*}

Étant donné que les vecteurs $\bar{\mu}$ et $\bar{\sigma}$ sont essentiellement composé d'instances normales, on peut s'attendre que les vecteurs $\mu$ et $\sigma$ de nouvelles instances normales soient relativement près de ces vecteurs moyen. À l'inverse, on peut s'attendre que ces 2 mêmes vecteurs, pour des instances anormales, ou des anomalies, soient plutôt loin de ces vecteurs moyens. Maintenant, comment savoir si la représentation latente d'une nouvelle instance est proche ou éloignée de cette représentation agrégée? Étant donné que les vecteurs $\mu$ et $\sigma$ représentent les paramètres d'une loi normale multivariée, nous pouvons utiliser la distance de Kullbach-Leibler pour quantifier le degré de similitude entre la distribution moyenne et la distribution $\mu$ et $\sigma$ de la nouvelle instance. Cette même distance est d'ailleurs utilisée lors de l'optimisation pour calculer la composante de perte associée à la représentation latente. Finalement, il est possible d'utiliser les distances de KL calculées sur l'ensemble d'entraînement pour savoir si une nouvelle distance est élevée ou non. Cela devient possible étant donné que notre ensemble d'entraînement contient presque uniquement des observations normales, donc des distances de KL qui devraient être petites. On peut d'ailleurs utiliser ces distances ordonnées pour avoir une mesure de probabilité d'une nouvelle instance. La méthodologie proposée pour calculer cette mesure de probabilité, ou cette valeur-$p$ est résumé dans l'algorithme \ref{anomaly_algo}.
\newline


\begin{center}
	\begin{algorithm}[H] \label{anomaly_algo}
		\SetAlgoLined
		\KwIn{Ensemble de données d'entraînement sans anomalie $x^{(j)}, j=1,...,m$, \\ Ensemble de données de test avec anomalies $y^{(i)}, i=1, ..., n$}
		\KwOut{Valeurs-$p$ pour chaque instance de test $p^{(i)}, i=1,...,n$}
		$\theta$, $\phi$ $\leftarrow$ paramètres de l'encodeur ($q_{\theta}(x)$) et du  décodeur ($p_{\phi}(z)$) du VAE entraîné\;
		\For{j=1 to m}{
			$\mu^{(j)} = p_{\theta}(x^{(j)})["mu"]$\;
			$\sigma^{(j)} = p_{\theta}(x^{(j)})["sd"]$\;
		}
		$\bar{\mu} = \frac{1}{m} \sum_{k=1}^{m} \mu^{(k)}$\;
		$\bar{\sigma} = \frac{1}{m} \sum_{k=1}^{m} \sigma^{(k)}$\;
		\For{j=1 to m}{
			$kl\_train^{(j)}=kl\_distance(\mu^{(j)}, \sigma^{(j)}, \bar{\mu}, \bar{\sigma})$
		}
		$kl\_sorted = sort(kl\_train)$\;
		\For{i=1 to n}{
			$\mu^{(i)} = p_{\theta}(y^{(i)})["mu"]$\;
			$\sigma^{(i)} = p_{\theta}(y^{(i)})["sd"]$\;
			$kl\_test^{(j)}=kl\_distance(\mu^{(i)}, \sigma^{(i)}, \bar{\mu}, \bar{\sigma})$\;
			$p^{(i)} = rank_{kl\_sorted}(kl\_test)$ / m
		}
		\KwRet{$\boldsymbol{p}$}
		\caption{Algorithme de détection d'anomalies basé sur un autoencodeur variationnel}
	\end{algorithm}
\end{center}

Dans l'approche que nous proposons, la valeur-$p$ est calculée à partir d'une distribution empirique, soit les distances de Kullbach-Leibler de toutes les instances du jeu de données d'entraînement. Au final, nous testons empiriquement si une nouvelle observation semble provenir de la population d'entraînement:

\begin{center}
	$\boldsymbol{H_0}$: $y^{(i)}$ provient de la population $X$ \\
	$\boldsymbol{H_1}$: $y^{(i)}$ ne provient pas de la population $X$
\end{center}

\noindent Puisque nous supposons que $X$, soit l'ensemble de données d'entraînement, est composé en grande partie d'observations normales, nous pouvons réécrire notre test comme étant:

\begin{center}
	$\boldsymbol{H_0}$: $y^{(i)}$ n'est pas une anomalie \\
	$\boldsymbol{H_1}$: $y^{(i)}$ est une anomalie
\end{center}

Finalement, une fois que nous avons notre test statistique de définit, nous pouvons utiliser les valeurs-$p$ calculées sur l'ensemble de test et conclure avec un niveau de confiance $\alpha$ si une instance est une anomalie (voir algorithme \ref{test_algo}).

\begin{center}
	\begin{algorithm}[H] \label{test_algo}
		\SetAlgoLined
		\KwIn{Valeurs-$p$ pour les instances de test  $p^{(i)}, i=1,...,n$, \\ niveau de confiance $\alpha$}
		\KwOut{indicateurs d'anomalies $o^{(i)}, i=1,...,n$}
		\For{i=1 to n}{
			\eIf{$p^{(i)} < \alpha$}{
				$o^{(i)}$ = $vrai$
			}
			{
				$o^{(i)}$ = $faux$
			}
		}
		\KwRet{$\boldsymbol{o}$}
		\caption{Algorithme de prise de décision}
	\end{algorithm}
\end{center}

\section{Approche pour des données complexes}

Un des objectifs de notre approche est de faire la détection d'anomalies parmi des données qui sont complexes, comme par exemple des images réelles. Sachant ce niveau de complexité, il faut que la méthodologie soit adaptée à ce contexte. Tout d'abord, il est pertinent de mentionner que le choix  d'utiliser des réseaux de neurones, plus particulièrement des autoencodeurs, est étroitement liée à ce concept de complexité. En effet, les réseaux de neurones ont le potentiel de stocker beaucoup d'informations au travers des nombreux paramètres du réseaux. De plus, les réseaux à convolutions sont également bien adaptés aux domaine de l'imagerie en prenant en compte l'information de manière locale dans l'image. Dans la prochaines sous-sections, nous allons présenter d'autres spécifications de l'approche qui permettent de prendre en compte cette complexité.

\subsection{Représentation latentes}



\subsection{Perceptual loss}

\section{Avantages du test d'hypothèse}

Parler du fait que c'est simple de definir un level of significance versus trouver une metric quelconque.             % methodologie
\chapter{Expérimentations}     % numéroté
\label{chap:experiments}                   % étiquette pour renvois (à compléter!)

\section{Jeux de données}

\subsection{MNIST}

\subsection{ImageNet}

\section{Modèles testés}

\section{Détails d'entraînement}

\section{Discussion}             % experimentations
\chapter*{Conclusion}           % ne pas numéroter
\label{chap:conclusion}         % étiquette pour renvois
\phantomsection\addcontentsline{toc}{chapter}{\nameref{chap:conclusion}} % inclure dans TdM

La détection d'anomalies est une tâche qui comporte généralement plusieurs défis en modélisation et en analyse de données. Parmi ces défis, on peut mentionner le débalancement entre la classe "normale" et "anormale". Ce débalancement peut s'expliquer par le fait qu'une anomalie est par définition un événement qui diffère significativement de la majorité des données \citep{Zimek2017} et ainsi peut probable. Ensuite, la détection d'anomalies nous amène généralement dans un contexte d'apprentissage non-supervisé. En effet, il est rare qu'on possède des étiquettes nous indiquant si une observation est "anormale" ou pas. Un défi important relié à ce contexte d'apprentissage non-supervisé est que l'analyste ou le chercheur se retrouve souvent dans une situation où il doit déterminer un seuil quelconque afin de pouvoir être en mesure discriminer les observations "normales" des observations "anormales". L'établissement de ce seuil est généralement basé sur une mesure de distance qui peut être difficile à établir dans certains cas et peut également différer d'une expérience à une autre. Ces défis reliés à la détection d'anomalies deviennent encore plus importants lorsqu'on doit traiter des données complexes, comme par exemple des images. Cette complexité supplémentaire peut se traduire par une forte dimensionnalité des données d'entrées et aussi par une notion de dépendance et de localité entre les pixels voisins d'une image.  Dans ce mémoire, nous démontrons la pertinence d'utiliser des autoencodeurs variationnels pour adresser les défis traditionnelles reliés à la détection d'anomalies tout en restant efficace dans une contexte de données complexes comme des images. En bout de ligne, notre approche permet d'identifier les images d'un jeu de données qui dévient de la normalité par rapport à un niveau de filtration, ou d'acceptabilité, intuitif et simple à établir.

Dans les expérimentations réalisées et présentées dans le chapitre \ref{chap:experiments}, nous sommes en mesure de conclure que notre approche produit des résultats performants autant sur des jeux de données d'images réelles que d'images simples. La valeur ajoutée de notre approche par rapport à d'autres méthodes est toutefois beaucoup plus évidente dans le cas d'images réelles. En effet, les performances obtenues par notre approche sur des images simples, comme \textit{MNIST}, ne sont généralement pas significativement meilleures ou pires que d'autres approches. En revanche, les performances en aire sous la courbe ROC, en précision et en rappel obtenues sur des images réelles plus complexes, comme \textit{ImageNet}, sont supérieures aux autres approches comparatives. Nous avons remarqué que l'autoencodeur variationnel permet de représenter les données complexes dans une forme latente beaucoup plus simple et qui permet aussi de bien discriminer les anomalies des observations "normales". 

Notre approche pourrait être utilisée pour plusieurs applications ou situations réelles où l'on cherche à identifier des images "anormales". Prenons l'exemple où l'on bénéficie d'un ensemble d'images propres pour lequel on sait qu'il n'y a pas d'anomalies ou de défectuosités. Supposons que dans le futur, on prévoit recevoir de nouvelles images qui sont supposées représenter la même chose que notre ensemble d'images propres, mais dont certaines de ces nouvelles images sont défectueuses où représentent quelque chose de complètement différent. Nous ne connaissons pas d'avance à quoi ces images défectueuses peuvent ressembler et n'avons aucun exemple d'image de la sorte sur lesquelles on pourrait faire un apprentissage. Notre approche permettrait donc de déterminer lesquelles de ces nouvelles images sont potentiellement défectueuses ou ne semblent pas représenter la même chose que notre ensemble d'images a priori. On pourrait également penser à des exemples d'applications où l'on cherche à nettoyer un jeu de données d'images. En effet, on pourrait valider et confirmer un sous-ensemble d'images propres qui représentent le jeu de données que l'on souhaite avoir. Ensuite, on pourrait utiliser ce sous-ensemble comme jeu de données d'entraînement pour notre approche. Finalement, on pourrait utiliser le reste du jeu de données initial comme jeu de données de test et ainsi valider si des images "anormales" font parties de notre ensemble global. Cela reviendrait donc utiliser notre approche comme technique de nettoyage de données sans avoir à valider toutes les images manuellement.

Dans le futur, il serait intéressant de voir si des variations de l'autoencodeur variationnel pourraient donner des résultats encore plus performants en détection d'anomalies. Parmi les possibilités de variations, il serait intéressant de voir si une autre loi que la $N(0,I)$ pourrait être utilisée comme loi a priori de la représentation latente et voir si cela permettrait d'obtenir une représentation latente qui discrimine encore mieux les anomalies des observations "normales". Ensuite, il serait également intéressant de voir s'il est possible de faire en sorte que les représentations latentes des anomalies se comportent toujours de la même manière par rapport à un point de référence, soit la distribution $N(0,I)$ dans notre cas. En effet, nous avons remarqué que selon la nature des données, les représentations latentes des anomalies se retrouvaient dans certains cas plus près de la $N(0,I)$ et dans certains cas  plus éloignées de la $N(0,I)$. Ce constat ajoute une étape manuelle pour mettre en place notre méthodologie, ce qui rend l'approche moins autonome et peut-être plus difficile à appliquer dans la pratique. Au final, nous pouvons conclure que l'autoencodeur variationnel possède des propriétés intéressantes, dont l'apprentissage d'une représentation latente utile, pour faire de la détection d'anomalies sur des données complexes comme des images.


            % conclusion

\appendix                       % annexes le cas échéant

%\include{annexe}                % annexe A

\bibliography{bibliography}                 % production de la bibliographie

\end{document}
